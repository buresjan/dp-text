\section{Non-dimensionalization and discretization}
The computational domain \( (0 ; L_1) \times(0 ; L_2) \times(0 ; L_3)\), where \( L_i\) \si{[m]}, $i=1,2,3$, represent the dimensions of the domain, is discretized using an equidistant grid with a spatial step size \( \Delta l \) \si{[m]}. The time interval is discretized uniformly using a time step size \( \Delta t \)~\si{[s]}.

In LBM, it is common to work with non-dimensional quantities instead of physical ones, as discussed in \cite{Kruger}. This section introduces the conversion relationships for transitioning to a non-dimensional system, defines the lattice used for domain discretization, and describes the discretized Boltzmann transport equation.

\subsection{Non-dimensional units}
The transition between unit systems can be performed using the law of similarity for fluid dynamics, which ensures that the values of relevant non-dimensional quantities remain constant \cite{Kruger}. One such non-dimensional quantity that can be used during the transition is, for example, the Reynolds number \eqref{Re}. It should be emphasized that the physical principles remain valid regardless of the choice of the unit system. 

In the following conversion relations, all quantities in lattice units are marked with the superscript \( L \). It can be shown \cite{Kruger} that the following holds:
\begin{eqnarray}
	l^{L}_0 &=& \dfrac{l_{0}}{\Delta l},\\[5pt]
	t^{L}_0 &=& \dfrac{t_{0}}{\Delta t},\\[5pt]
	\nu^L &=& \nu \dfrac{\Delta t}{\Delta l^{2}},\\[5pt]
	u^{L}_{i} &=& \dfrac{\Delta t}{\Delta l} u_{i}, \ i \in \{1, 2, 3\}.
\end{eqnarray}
The characteristic length, time, and velocity values are chosen based on the given physical problem. The computational domain's largest dimension or the size of an obstacle within the flow is typically chosen as \( l_{0} \). Detailed derivations of these relations can be found in \cite{Kruger}. From the relationship for kinematic viscosity, it can be seen that for a given spatial step \( \Delta l \), the time step \( \Delta t \) is linked to the value of \( \nu^L \).

For simplicity, in this work, we will assume that the spatial step \( \Delta l ^L \) and the time step \( \Delta t ^L \) in lattice units are \( \Delta l ^L  =  \Delta t ^L = 1 \). In the remainder of this chapter, we use lattice units exclusively and for brevity we omit the superscript \( L \), though all quantities will be non-dimensional.