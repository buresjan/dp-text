
In this section, we summarize the additional assumptions applied to our system to simplify the governing equations \eqref{NS} and focus on the modeling of the TCPC.

Specifically, the model assumes an isothermal system (i.e., constant temperature over time) with no external forces acting on it. Blood is treated as an incompressible, Newtonian fluid with constant dynamic viscosity. This assumption is justified as the model of TCPC involves relatively large vessels where the non-Newtonian effects of blood, such as viscosity variations, are expected to be negligible.

The vessel walls are assumed to be rigid, ignoring their elasticity. While this simplification reduces the model's physiological fidelity, it significantly decreases the computational complexity of the simulations. 
Studies such as \cite{Masters2004} and \cite{Orlando2006} have shown that neglecting compliance in larger vessels can impact the accuracy of results, particularly in terms of power loss and flow distribution. However, correctly modeling fluid-structure interactions to incorporate wall compliance would require detailed \textit{in vivo} data and significantly increase the computational demand.

Additionally, we assume a constant, steady inflow profile, neglecting pulsatile flow variations typically observed in physiological blood flow. While pulsatility can influence flow patterns and energy dissipation, its omission simplifies the boundary conditions. This assumption is particularly suited to studying time-averaged hemodynamic metrics, which are often of primary interest in TCPC modeling.

We consider a rectangular domain $ \Omega \subset \mathbb{R}^3 $, defined by $ \Omega = (0, L_1) \times (0, L_2) \times (0, L_3) $, where $ L_1 $, $ L_2 $, and $ L_3 $ \si{[m]} represent the domain dimensions along each spatial axis. The temporal interval is denoted by $ \mathcal{I} = \langle 0, T \rangle $, where $ T > 0 $.
\newpage
With these assumptions, the governing equations \eqref{NS} on $ \Omega \times \mathcal{I} $ simplify to \cite{Schlichting}:

\begin{subequations}\label{NS s predpoklady}
	\begin{gather}
		\label{a s predpoklady}
		\nabla \cdot \vec{u} = 0, \\[5pt]
		\label{b s predpoklady}
		\rho \frac{\text{D} \vec{u}}{\text{D} t} = - \nabla p + \mu \Delta \vec{u},
	\end{gather}
\end{subequations}
where we have introduced the material derivative operator
\begin{equation}
	\dfrac{\text{D}}{\text{D} t} \coloneqq \dfrac{\partial}{\partial t} + \vec{u} \cdot \nabla.
\end{equation}

The boundary of $ \Omega $ is split into parts as follows: $ \partial \Omega_{\text{in}} $ denotes the inflow boundary where inlet conditions are specified; $ \partial \Omega_{\text{out}} $ represents the outflow boundary; and $ \partial \Omega_{\text{w}} $ corresponds to the wall boundaries, where the no-slip condition is imposed. The system \eqref{NS s predpoklady} is supplemented by the following initial and boundary conditions:

\begin{subequations}\label{eq:okrajovky}
	\begin{alignat}{3}
		&\vec{u} = \vec{u}_{\text{ini}},  &p = p_{\text{ini}} \hspace{5mm} &\text{on } \ \Omega \times \mathcal{I},\\[3pt]
		&(\nabla p - \nu \Delta \vec{u}) \cdot \vec{n}  = 0, &\vec{u} = \vec{u}_{\text{in}} \hspace{5mm} &\text{on } \ \partial \Omega_{\text{in}} \times \mathcal{I},\\[3pt]
		&\vec{u} = \vec{0}, \, &\nabla p \cdot \vec{n} = 0 \hspace{5mm} &\text{on } \ \partial \Omega_{\text{w}} \times \mathcal{I},\\[3pt]
		&p = p_{\text{out}}, &\nabla u_i \cdot \vec{n} = 0 \hspace{5mm} &\text{on } \ \partial \Omega_{\text{out}} \times \mathcal{I}, \quad i = 1, 2, 3,
	\end{alignat}
\end{subequations}
where $ \vec{n} $ is the outward-pointing normal unit vector to the boundary $ \partial \Omega $. Here, $ \vec{u}_{\text{ini}}$ \si{[m.s^{-1}]} and $ p_{\text{ini}} $ \si{[kg.m^{-1}.s^{-2}]} denote the initial velocity and pressure, respectively; $ \vec{u}_{\text{in}}$ \si{[m.s^{-1}]} is the prescribed inflow velocity at $ \partial \Omega_{\text{in}} $; and $ p_{\text{out}} $~\si{[kg.m^{-1}.s^{-2}]} is the prescribed outflow pressure at $ \partial \Omega_{\text{out}} $. Initial and boundary conditions are further discussed in Section~\ref{pocatecni a okrajove podminky}.

