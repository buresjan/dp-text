\section{Introduction}

\subsection{Purpose of Geometry Generation}

The geometry generation process is a crucial component of the optimization framework, as it provides the physical representation of the system to be optimized. In our case, the geometry is directly influenced by the optimization parameters, which control aspects such as dimensions, resolution, and mesh density. These parameters allow for the dynamic generation of geometries that accurately represent the system, enabling precise numerical simulations and effective optimization.

\subsection{Overview of the Process}

The process begins with a set of optimization parameters that dynamically modify predefined geometry templates. These templates, written in Gmsh scripting language, form the foundation for generating the geometry. After parameterization, the geometry is meshed and exported in STL format. Subsequently, the STL file is processed using Trimesh to create a rectilinear mesh suitable for numerical simulations. 

This entire workflow is automated through a custom Python package, which integrates template modification, mesh generation, and post-processing in a seamless pipeline.

% [Figure suggestion: A diagram showing the workflow, starting with optimization parameters, followed by modifying Gmsh templates, exporting STL, and processing with Trimesh to obtain a rectilinear mesh suitable for simulations.]

