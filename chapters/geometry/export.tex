\section{Exporting the Geometry}

\subsection{Output Format}

Once the voxelized mesh has been generated, it is saved in a format suitable for numerical simulations. The chosen output format is a 3D NumPy array, which efficiently stores the voxel data and can be easily manipulated and processed in Python. This format is particularly useful for simulations that require structured data, such as fluid dynamics simulations using the Lattice Boltzmann Method (LBM).

The voxelized geometry is saved in `.npy` files, which is a binary format for storing NumPy arrays. This ensures that the mesh can be loaded quickly into memory with minimal overhead. Additionally, the option to export the mesh as a plain text file is available, where each line contains the coordinates of a voxel and its corresponding value (e.g., 1 for solid, 0 for void). This text format provides transparency and compatibility with a wide range of tools, though it is less efficient in terms of storage and loading time.

% [Figure suggestion: A small illustration showing the 3D voxel grid and the corresponding exported formats (.npy and text file).]

\subsection{Integration with Simulation Frameworks}

The final exported geometry is designed to integrate seamlessly with numerical simulation frameworks. By converting the STL files into voxelized grids, we ensure that the output is compatible with methods like LBM, which rely on structured grids. The voxel data can be directly fed into the simulation code, where it serves as the computational domain for fluid or heat flow simulations.

An advantage of using voxelized meshes is that they allow for easy manipulation during the simulation process. For example, specific regions of the mesh can be modified or refined without requiring a complete regeneration of the geometry. This flexibility is critical for iterative optimization processes where the geometry needs to be adjusted based on simulation results.

