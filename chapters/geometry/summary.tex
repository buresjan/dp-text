\section{Conclusion}

In this chapter, we outlined the entire geometry generation process, which forms an integral part of the optimization framework. By leveraging parameterized Gmsh templates and utilizing tools like Trimesh for mesh processing, we can generate and customize complex geometries dynamically based on optimization parameters. 

The Python package developed for this purpose automates the workflow, allowing the user to generate voxelized meshes efficiently. Key functionalities of the package include reading and modifying Gmsh templates, exporting geometries in STL format, voxelizing meshes, and integrating the final outputs with numerical simulation frameworks.

The flexibility of this approach ensures that the geometry generation process can accommodate a wide range of scenarios, making it adaptable to various types of optimization problems. The modular design of the package also makes it easy to extend or modify specific components without affecting the entire system.

Looking ahead, there are several opportunities for future improvements. These could include adding support for more complex geometries, refining the voxelization process for greater accuracy, or exploring alternative meshing techniques that could provide better performance in certain simulation scenarios. 

Overall, the tools and processes described in this chapter contribute significantly to the overall optimization framework, providing a reliable and adaptable solution for generating geometries that are essential to the simulation and optimization workflow.

