\chapter{Custom Nelder-Mead algorithm implementation}\label{appendix C}

\section{Key features}

\subsection*{Initialization}
The simplex is initialized around the starting point \texttt{x\_start} using a defined step size. The function values for each vertex of the simplex are evaluated to initialize scores.

\subsection*{Stopping criteria}
The algorithm stops when:
\begin{itemize}
	\item The maximum number of iterations (\texttt{max\_iter}) is reached.
	\item The objective function improvement over consecutive iterations is below a defined threshold\\ (\texttt{no\_improve\_thr}) for a specified number of iterations (\texttt{no\_improv\_break}).
\end{itemize}

\subsection*{Function evaluations}
While the implementation is suboptimal in terms of the total number function evaluations, this design choice was made because, in the case specific for this work, a single function evaluation usually takes up to several hours. Thus, the algorithm prioritizes evaluating all candidate points (with potential parallelization) before making comparisons or updates to the simplex. This approach ensures that computational resources are utilized efficiently, even at the cost of additional function evaluations in some scenarios.

\section{The implementation}

The main part of the custom implementation is presented in Listing~\ref{lst:NM}. Full implementation is available upon request on Github at \href{https://github.com/buresjan/nelder\_mead}{\texttt{https://github.com/buresjan/nelder\_mead}}. The online code also includes several test cases showing the functionality of the custom implementation.

\newpage
\begin{lstlisting}[
	language=Python,
	caption={Main part of the custom Nelder-Mead algorithm implementation.  Full implementation is available on Github at \href{https://github.com/buresjan/nelder\_mead}{\texttt{https://github.com/buresjan/nelder\_mead}}.},
	label={lst:NM}]
def nelder_mead(
	f,
	x_start,
	step=0.01,
	no_improve_thr=1e-6,
	no_improv_break=20,
	max_iter=1000,
	# Standard Nelder-Mead parameters
	delta_e=2.0,  # expansion coefficient
	delta_oc=0.5,  # outside contraction coefficient
	delta_ic=0.5,  # inside contraction coefficient
	gamma=0.5,  # shrink coefficient
	verbose=False,
):
simplex, scores = initialize_simplex(f, x_start, step, verbose)
prev_best = scores[0]
no_improv = 0
iter_count = 0

while True:
	# Order simplex by score
	simplex, scores = order_simplex(simplex, scores)
	best = scores[0]

	# Verbose output
	if verbose:
		print(
			f"Iteration {iter_count}: Best estimate = {simplex[0]}, "
			f"Function value = {best}, "
			f"Simplex: {simplex}"
		)

	# Check stopping criteria
	if max_iter and iter_count >= max_iter:
		return simplex[0], best
	if no_improv >= no_improv_break:
		return simplex[0], best

	iter_count += 1
	if best < prev_best - no_improve_thr:
		prev_best = best
		no_improv = 0
	else:
		no_improv += 1
	
	# Compute centroid (excluding the worst point)
	centroid = compute_centroid(simplex)
	
	# Generate candidate points
	candidates = generate_candidate_points(
		centroid, simplex[-1], delta_e, delta_oc, delta_ic
	)
	
	# Evaluate candidates in parallel
	candidate_keys = [
		"reflection",
		"expansion",
		"outside_contraction",
		"inside_contraction",
	]
	candidate_values = [candidates[key] for key in candidate_keys]
	
	with concurrent.futures.ProcessPoolExecutor() as executor:
		future_to_key = {
			executor.submit(f, candidate): key
			for key, candidate in zip(candidate_keys, candidate_values)
		}
		results = {}
		for future in concurrent.futures.as_completed(future_to_key):
			key = future_to_key[future]
			results[key] = future.result()
	
	f_r = results["reflection"]
	f_e = results["expansion"]
	f_oc = results["outside_contraction"]
	f_ic = results["inside_contraction"]
	
	# Nelder-Mead logic:
	# 1. If reflection is better than the best point, try expansion
	if f_r < scores[0]:
		if f_e < f_r:
			# Expansion is better than reflection
			simplex[-1] = candidates["expansion"]
			scores[-1] = f_e
			continue
		else:
			# Keep reflection
			simplex[-1] = candidates["reflection"]
			scores[-1] = f_r
			continue
	
	# 2. If reflection is not better than best, but better than second-worst, accept reflection
	elif f_r < scores[-2]:
		simplex[-1] = candidates["reflection"]
		scores[-1] = f_r
		continue
	
	# 3. If reflection is worse or equal to second-worst but better than worst, do outside contraction
	elif f_r < scores[-1]:
		if f_oc <= f_r:
			# Outside contraction improved over worst
			simplex[-1] = candidates["outside_contraction"]
			scores[-1] = f_oc
			continue
		else:
			# Outside contraction didn't improve, shrink
			simplex, scores = shrink_simplex(simplex, scores, gamma, f)
			continue
	
	# 4. Reflection is not better than worst, try inside contraction
		else:
		if f_ic < scores[-1]:
			# Inside contraction improved over worst
			simplex[-1] = candidates["inside_contraction"]
			scores[-1] = f_ic
			continue
		else:
			# Inside contraction didn't improve, shrink
			simplex, scores = shrink_simplex(simplex, scores, gamma, f)
			continue
\end{lstlisting}