%~\newpage{}

\selectlanguage{czech}%
\begin{onehalfspace}
\noindent \emph{Název práce:}
\noindent \textbf{Optimální tvar stěn idealizovaného kavopulmonálního spojení}
\end{onehalfspace}
\noindent \emph{Autor:} Bc. Jan Bureš\\[2pt]
\noindent \emph{Program:} Matematické inženýrství\\[2pt]
\noindent \emph{Druh práce:} Diplomová práce\\[2pt]
\noindent \emph{Vedoucí práce:} doc. Ing. Radek Fučík, Ph.D.,
Katedra matematiky a katedra softwarového inženýrství, FJFI ČVUT v Praze
Trojanova 13, 120 00 Praha\\[2pt]
\noindent \emph{Konzultant:} MUDr. Mgr. Radomír Chabiniok, Ph.D., University of Texas Southwestern Medical Center, USA\\[2pt]
\noindent \emph{Abstrakt:} Tato práce se zabývá optimalizací tvaru idealizovaného modelu úplného kavopulmonální spojení (TCPC)--chirurgického zákroku prováděného u pacientů s vrozenou srdečnou vadou zahrnující funkčně jednou komoru. Byl vyvinut optimalizační rámec, který zahrnuje generování geometrie v jazyce Python, simulaci pomocí mřížkové Boltzmannovy metody (LBM) a bezgradientní optimalizační algoritmy -- Nelderovu-Meadovu metodu a síťové adaptivní přímé vyhledávání. Vyvinutý rámec automatizuje proces generování parametrizovaných 3D geometrií, simulace proudění nestlačitelné newtonovské tekutiny a vyhodnocování účelových funkcí. Navržený rámec byl testován na zjednodušených modelech TCPC s různými optimalizačními parametry. Výsledky prokázaly funkčnost a účinnost navrženého přístupu. Přestože se studie zaměřuje pouze na idealizované geometrie se zjednodušenými předpoklady, jako je rigidita stěn cév a ustálené proudění, poskytuje základ pro rozšíření na data specifická pro pacienty a pro zahrnutí složitějších fyziologických podmínek. Tato práce představuje krok vpřed v aplikaci optimalizace v kardiochirurgii s potenciálem zlepšit klinické výsledky a plánování léčby specifické pro jednotivé pacienty.

\bigskip{}

\noindent \emph{Klíčová slova:} bezgradientní optimalizace, modelování proudění krve, mřížková Boltzmannova metoda, optimalizace tvarů, úplné kavopulmonární spojení
\vfill{}
~

\selectlanguage{american}%
\begin{onehalfspace}
\noindent \emph{Title:}
\noindent \textbf{Optimal wall geometry of an idealized total cavopulmonary connection}
\end{onehalfspace}
\noindent \emph{Author:} Bc. Jan Bureš\\[2pt]
\noindent \emph{Abstract:} This thesis addresses the optimization of wall geometry for an idealized Total Cavopulmonary Connection (TCPC), a surgical procedure used to treat congenital heart defects involving a single functional ventricle. A custom optimization framework was developed, integrating Python-based geometry generation, simulation using the Lattice Boltzmann method (LBM), and gradient-free optimization algorithms--Nelder-Mead and Mesh adaptive direct search methods. The framework automates the process of generating parameterized 3D geometries, simulating flow of incompressible Newtonian fluid, and evaluating objective functions. The proposed framework was tested on simplified TCPC models with varying optimization parameters. Results demonstrated the feasibility and effectiveness of the approach. While the study focuses on idealized geometries with simplified assumptions, such as rigid vessel walls and steady flow, it provides a robust foundation for extending the framework to patient-specific data and more complex physiological conditions. This work represents a step forward in applying computational optimization to cardiovascular surgery, with the potential to enhance clinical outcomes and patient-specific treatment planning.

\bigskip{}

\noindent \emph{Key words:} gradient-free optimization, lattice Boltzmann method, modeling of blood flow, shape optimization, total cavopulmonary connection
