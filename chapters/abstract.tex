%~\newpage{}

\selectlanguage{czech}%
\begin{onehalfspace}
\noindent \emph{Název práce:}
\noindent \textbf{Hledání optimálního tvaru stěn matematického modelu proudění krve v problematice úplného kavopulmonálního cévního napojení}
\end{onehalfspace}
\noindent \emph{Autor:} Bc. Jan Bureš\\[2pt]
\noindent \emph{Program:} Matematické inženýrství\\[2pt]
\noindent \emph{Druh práce:} Diplomováce\\[2pt]
\noindent \emph{Vedoucí práce:} doc. Ing. Radek Fučík, Ph.D.,
Katedra matematiky a katedra softwarového inženýrství, FJFI ČVUT v Praze
Trojanova 13, 120 00 Praha\\[2pt]
\noindent \emph{Konzultant:} Mudr. Mgr. Radomír Chabiniok, Ph.D., University of Texas Southwestern Medical Center, USA\\[2pt]
\noindent \emph{Abstrakt:} \lipsum[1]\lipsum[2]

\bigskip{}

\noindent \emph{Klíčová slova:} matematická optimalizace, modelování proudění krve, mřížková Boltzmannova metoda, optimalizace tvarů, úplné kavopulmonární spojení
\vfill{}
~

\selectlanguage{american}%
\begin{onehalfspace}
\noindent \emph{Title:}
\noindent \textbf{Optimal shape design of walls of blood flow mathematical model focusing on the total cavopulmonary connection}
\end{onehalfspace}
\noindent \emph{Author:} Bc. Jan Bureš\\[2pt]
\noindent \emph{Abstract:} \lipsum[1]

\bigskip{}

\noindent \emph{Key words:} mathematical optimization, modeling of blood flow, lattice Boltzmann method, shape optimization, total cavopulmonary connection
