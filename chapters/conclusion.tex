\chapter*{Conclusion}

\pagestyle{plain}

\addcontentsline{toc}{chapter}{Conclusion}

The primary objective of this thesis was to develop and evaluate an optimization framework tailored for improving the geometry of an idealized total cavopulmonary connection (TCPC). The framework integrates numerical simulations using the lattice Boltzmann method (LBM), automatic geometry generation, and gradient-free optimization techniques.

The thesis begins by presenting the mathematical model employed to describe blood flow in vessels, along with the assumptions and simplifications made to ensure computational feasibility. The second chapter introduces the LBM as the chosen numerical method. The third chapter outlines the custom geometry generation pipeline, emphasizing its capability to produce simulation-ready voxelized geometries from parameterized templates. In the fourth chapter, gradient-free optimization methods are discussed along with the design of the proposed optimization framework itself. Finally, the results of applying this framework to two simplified TCPC models are presented, validating its functionality and demonstrating its flexibility.

Building on previous work \cite{buresBP, buresVU}, this thesis extends the application of optimization to three-dimensional geometries and explores the behavior of simplified TCPC models. The study assumes various simplifications enabling efficient testing and validation of the framework. The results demonstrate the framework's ability to tackle complex problems, validating its potential to optimize surgical designs in the context of cardiovascular systems. The study further illustrates how computational fluid dynamics (CFD) can provide both quantitative evaluations and qualitative insights. It is illustrated how CFD can be used to provide detailed in-depth analyses, which could aid clinicians in making informed decisions. 

The insights gained from this study establish a solid foundation for future research. By extending the framework to account for more complex physiological settings and to model patient-specific geometries, it holds promise for addressing clinically relevant challenges. Collaborations with institutions such as Institute for Clinical and Experimental Medicine (IKEM) in Prague and UT Southwestern (UTSW) Medical Center in Dallas will provide opportunities to validate the model against experimental data, such as flow MRI, and apply it to real-world cases. These advancements could bridge the gap between theoretical research and practical implementation, paving the way for improved clinical outcomes and personalized treatment planning. This work thus represents a meaningful step forward in applying computational optimization to cardiovascular surgeries, while also underlining the potential of CFD in broader qualitative and quantitative analyses of complex cardiovascular systems.