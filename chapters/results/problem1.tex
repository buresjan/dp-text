\section{Problem with one optimization parameter}

\subsection{Problem setup}
\begin{problem}{Basic cylindrical junction}
	\vspace{2mm}
	Physical setup:
	\begin{itemize}
		\item Domain: $ \Omega=(0 ; 170 \mathrm{~mm}) \times(0 ; 22 \mathrm{~mm}) \times(0 ; 100 \mathrm{~mm})$.
		\item Kinematic viscosity: $ \nu=3 \cdot 10^{-6} \mathrm{~m}^{2} \mathrm{~s}^{-1}$.
		\item Inflow at $\Gamma^{\text{IVC}}_{\text{in}}$: a constant velocity of magnitude $0{,}45$ \si{m s^{-1}},  aligned with the IVC axis in the positive $x_3$-direction.
		\item Inflow at $\Gamma^{\text{SVC}}_{\text{in}}$: a constant velocity of magnitude $0{,}35$ \si{m s^{-1}}, aligned with the SVC axis in the negative $x_3$-direction.
	\end{itemize} 
	LBM setup:
	\begin{itemize}
		\item Initial condition on $ \overline{\hat{\Omega}} $ set as described in Section~\ref{pocatecni podminka}.
		\item Boundary conditions at $\Gamma^{\text{W}}_{\text{out}}$ and $\Gamma^{\text{E}}_{\text{out}}$ set according to Section~\ref{symmetric bc}.
		\item Discretization: $\overline{\hat{\Omega}} = N_{x} \times N_{y} \times N_{z}$, $N_{x} = 768, \, N_{y} = 105, N_{z} = 453$.
		\item Kinematic viscosity in lattice units: $\nu^{L} = 10^{-3} $.
	\end{itemize}
	Optimzation setup:
	\begin{itemize}
		\item Using Model 1, as described in Section~\ref{mod:model1} and illustrated in Figure~\ref{fig:model1_schematic}.
		\item Using the custom Nelder–Mead and MADS (NOMAD implementation) methods, both described in Chapter~\ref{optimization}
	\end{itemize}
	\label{prob:1}
\end{problem}
\newpage
\subsection{Results and discussion}


\begin{figure}[H]
	\subsubsection*{Mean velocities}
	{\centering
	\begin{subfigure}{0.48\textwidth}
		\centering
		\includegraphics[
			width=\textwidth,
			trim={0mm 0mm 0mm 0mm}
		]
			{figures/plots/08/08_mean_veloc_XZ-c.pdf}
	\end{subfigure}\hfill%
	\begin{subfigure}{0.48\textwidth}
		\centering
		\includegraphics[
		width=\textwidth,
		trim={0mm 0mm 0mm 0mm}
		]
		{figures/plots/08/08_mean_veloc_XZ-c.pdf}
	\end{subfigure}
	\begin{center}
			\includegraphics[
		width=0.4\textwidth,
		trim={0mm 0mm 0mm 0mm}
		]
		{figures/plots/08/veloc_legend.pdf}
	\end{center}
	}

%	\begin{subfigure}{0.24\textwidth}
%		\centering
%		\includegraphics[
%		width=0.8\textwidth,
%		trim={0mm 0mm 0mm 0mm}
%		]
%		{figures/plots/08/08_LPA_veloc-c.pdf}
%		\caption{}
%	\end{subfigure}\hfill%
%	\begin{subfigure}{0.24\textwidth}
%		\centering
%		\includegraphics[
%		width=0.8\textwidth,
%		trim={0mm 0mm 0mm 0mm}
%		]
%		{figures/plots/08/08_RPA_veloc-c.pdf}
%		\caption{}
%	\end{subfigure}
%	\begin{subfigure}{0.24\textwidth}
%		\centering
%		\includegraphics[
%		width=0.8\textwidth,
%		trim={0mm 0mm 0mm 0mm}
%		]
%		{figures/plots/08/08_LPA_angle-c.pdf}
%		\caption{}
%	\end{subfigure}
%	\begin{subfigure}{0.24\textwidth}
%		\centering
%		\includegraphics[
%		width=0.8\textwidth,
%		trim={0mm 0mm 0mm 0mm}
%		]
%		{figures/plots/08/08_RPA_angle-c.pdf}
%		\caption{}
%	\end{subfigure}
%	}
%	\subsubsection*{Model 1: Simplified cylindrical junction}
%	\begin{subfigure}{1.0\textwidth}
%		\centering
%		\includegraphics[
%		width=1.0\textwidth,
%		trim={60mm 10mm 60mm 0mm}
%		]
%		{figures/plots/08/08_mean_flucs_3D.pdf}
%		\caption{}
%	\end{subfigure}\hfill%
	\vspace{2mm}
	\caption{}
	\label{fig:}

\end{figure}

