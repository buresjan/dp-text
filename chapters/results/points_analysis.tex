\subsection*{Analysis of key points}
\noindent
Next, we summarize key observations for the studied points:
\subsubsection*{Point P\(_0\) \(\bigl(o_1 = 0.0\,\mathrm{cm}\bigr)\)}
\begin{itemize}
	\item As evident in Figure~\ref{fig:svc_ivc_split00}, the IVC and SVC flows split nearly 50--50\% between the LPA and RPA. 
	\item Streamlines in Figure~\ref{fig:mean_velocity_streamlines00} remain closely parallel to the PA axis at the outflows.
\end{itemize}
\subsubsection*{Point P\(_7\) \(\bigl(o_1 = 0.7\,\mathrm{cm}\bigr)\)}
\begin{itemize}
	\item Figure~\ref{fig:mean_velocity_streamlines07} illustrates the emergence of helical flow patterns, particularly into the RPA. Figure~\ref{fig:svc_ivc_split07} shows that SVC flow is pushed closer to the vessel wall, enhancing the helical pattern.
	\item Offsetting by roughly one vessel radius intensifies vortex structures, contributing to higher turbulent kinetic energy.
\end{itemize}

\subsubsection*{Point P\(_8\) \(\bigl(o_1 = 0.8\,\mathrm{cm}\bigr)\)}
\begin{itemize}
	\item Figure~\ref{fig:mean_velocity_streamlines08} illustrates that the helical flow remains but is less intense than at P\(_7\).
	\item While this configuration corresponds to a local minimum of $\dot{\gamma}^{A}_{\text{nw}}$ at, fluctuations are still prominent as depicted in Figure~\ref{fig:velocity_fluctuations08}.
	\todo[inline]{Add notes about the blood clogging and thrombotic implications.}
\end{itemize}

\subsubsection*{Point P\(_{24}\) \(\bigl(o_1 = 2.4\,\mathrm{cm}\bigr)\)}
\begin{itemize}
	\item At large offsets, helical flow is not as prominent, and flow near the outflows aligns more closely with the PA axis as illustrated if Figure~\ref{fig:mean_velocity_streamlines24} and Figure~\ref{fig:velocity_angle_outlets24}.
	\item As shown in Figure~\ref{fig:svc_ivc_split24}, IVC flow effectively completely pushes the SVC flow into the LPA, leading to highly unbalanced flow distribution which is physiologically suboptimal.
\end{itemize}

In our idealized system, a symmetric (zero-offset) geometry balances the two inflows, reducing turbulent effects and distributing the near-wall shear rate evenly. While this configuration could be considered locally (in terms of local minima) favorable for this particular studied idealized system, it is not geometrically feasible in realistic TCPC designs. Previous studies have likewise reported beneficial flow patterns in idealized TCPC models \cite{Masters2004, Dubini1996}. However, real physiological data indicate that in real physiology zero offset often corresponds to increased turbulence and higher energy dissipation \cite{Sharma1996, Rijnberg2018, DeGroff2007, Amodeo2004}. 

Overall, the obtained results align with other studies also reporting significant flow changes for offsets equal to half the IVC diameter and highlighting the preference  for offsets of approximately equal to one IVC diameter. \cite{Sharma1996, Rijnberg2018, Pekkan2005}. 

Finally, note that the anastomosis in the studied idealized system in this section lacks complex geometrical features such as flaring or curving, thus, the model serves mainly for the purposes of validating the optimization framework.
