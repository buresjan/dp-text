%~\newpage{}
\begin{onehalfspace}
\noindent \emph{Název práce:}
\noindent \textbf{Hledání optimálního tvaru stěn matematického modelu proudění krve v problematice úplného kavopulmonálního cévního napojení}
\end{onehalfspace}
\noindent \emph{Autor:} Bc. Jan Bureš\\[2pt]
\noindent \emph{Program:} Matematické inženýrství\\[2pt]
\noindent \emph{Druh práce:} Výzkumný úkol\\[2pt]
\noindent \emph{Vedoucí práce:} doc. Ing. Radek Fučík, Ph.D.,
Katedra matematiky a katedra softwarového inženýrství, FJFI ČVUT v Praze
Trojanova 13, 120 00 Praha\\[2pt]
\noindent \emph{Konzultant:} Ing. Pavel Eichler, Katedra softwarového inženýrství, FJFI ČVUT v Praze
Trojanova 13, 120 00 Praha 2\\[2pt]
\noindent \emph{Abstrakt:} Tato práce se zabývá optimalizací tvaru stěn v rámci modelování proudění nestlačitelné newtonovské tekutiny se zaměřením na modelování proudění krve v cévách. Je představen a implementován optimalizační rámec, který lze následně využít pro řešení optimalizačních úloh týkajících se proudění tekutin okolo rigidních překážek ve 2D. Pro numerické řešení matematického modelu je zvolena mřížková Boltzmannova metoda, která je stručně popsána.
Na hranici obtékaných těles jsou předepsány interpolační okrajové podmínky, které jsou popsány a dále použity. Díky interpolačním okrajovým podmínkám je zohledněn skutečný tvar hranice těles.
V teoretické části jsou pak dále popsány metody matematické optimalizace použité v této práci. Dále je popsán balík využitý k automatickému generování geometrií použitelných v numerických simulací, který byl implementován pro účely této práce. Praktická část demonstruje a analyzuje použití optimalizačního rámce na sérii vhodně navržených testovacích úloh. Na závěr jsou prezentovány výsledky optimalizační úlohy zjednodušeného modelu totálního kavopulmonálního spojení ve 2D, které jsou ve shodě s dostupnou literaturou. Použití optimalizačního rámce lze tedy považovat za úspěšné.

\bigskip{}

\noindent \emph{Klíčová slova:} matematická optimalizace, modelování proudění krve, mřížková Boltzmannova metoda, optimalizace tvarů, úplné kavopulmonární spojení
\vfill{}
~

%Interpolation boundary conditions are prescribed on the boundary of the enveloped bodies, which are described and used below. Due to the interpolation boundary conditions, the actual shape of the boundary of the bodies is taken into account.

\selectlanguage{american}%
\begin{onehalfspace}
\noindent \emph{Title:}
\noindent \textbf{Optimal shape design of walls of blood flow mathematical model focusing on the total cavopulmonary connection}
\end{onehalfspace}
\noindent \emph{Author:} Bc. Jan Bureš\\[2pt]
\noindent \emph{Abstract:} This thesis deals with the optimization of shape of walls and with flow modelling of incompressible Newtonian fluid with a focus on modelling of blood flow in blood vessels. An optimization framework is presented and implemented, which can then be used to solve optimization problems involving fluid flow around rigid objects in 2D. The lattice Boltzmann method is used as the numerical solver and is briefly described. The theoretical section then describes the mathematical optimization methods used in this work. Interpolation boundary conditions, which are described and later used, are prescribed on the boundary of the objects. Thanks to the interpolation boundary conditions, the actual shape of the boundary of the objects is taken into account. Furthermore, the package used to automatically generate geometries used in the numerical simulations, which was implemented for the purpose of this work, is described. The next part demonstrates and analyses the application of the optimization framework on a series of test problems. Finally, the results of the optimization problem of a simplified 2D total cavopulmonary connection model are presented, which are in agreement with the available literature. Thus, the application of the optimization framework can be considered successful.

\bigskip{}

\noindent \emph{Key words:} mathematical optimization, modeling of blood flow, lattice Boltzmann method, shape optimization, total cavopulmonary connection
\selectlanguage{czech}%