\chapter*{Závěr}


\pagestyle{plain}

\addcontentsline{toc}{chapter}{Záv\v{e}r}

Mezi cíle této práce patřilo studium a sestavení matematického modelu proudění krve v cévách a následný návrh a dále pak zejména implementace a otestování funkčnosti optimalizačního rámce na vhodně zvolených testovacích optimalizačních úlohách ve 2D.

První kapitola je věnována základním vztahům popisujícím dynamiku tekutin, dále jsou zde představeny základní poznatky týkající se sestavení matematického modelu proudění krve a jeho možného zjednodušení za vhodných předpokladů. Na závěr této kapitoly jsou definovány předpoklady, které v rámci této práce klademe na námi použitý model. V druhé kapitole je čtenáři představena použitá numerická metoda, tj. mřížková Boltzmannova metoda (LBM). Jsou zde rozebrány základní aspekty této metody, různé typy volby okrajových podmínek a výpočet silového působení v rámci LBM. Zejména je popsána důležitost použití interpolčaních okrajových na hranici obtékaných těles, díky nimž lze hranici neschodovitě diskretizovat a zohlednit tak její skutečný tvar. V třetí kapitole je detailně popsán balík implementovaný v programovacím jazyce Python sloužící ke generování různých objektů, které lze následně využít v numerických simulacích. Čtvrtá kapitola se věnuje teorii matematického programování a shrnutí optimalizačních metod použitých v této práci. Na závěr této kapitoly je představen navržený optimalizační rámec.

V páté kapitole jsou postupně společně s jejich výsledky představeny všechny navržené úlohy na nichž je optimalizační rámec testován. Nejdříve je ověřena jeho funkčnost na testovacích úlohách s jedním optimalizačním parametrem, dále je pak formulována úloha s dvěma optimalizačními parametry. Je mimo jiné testován vliv volby optimalizační metody s ohledem na rychlost a kvalitu řešení. Na závěr této kapitoly je představen zjednodušený parametrizovaný model cévní křižovatky vznikající při provedení úplného kavopulmonárního spojení a je pozorováno, jaký vliv na řešení má volba optimalizované účelové funkce. Funkčnost navrženého optimalizačního rámce byla na výše zmíněných úlohách prokázána.

Práce z části navazuje a rozšiřuje předchozí bakalářskou práci \cite{JB} věnující se zejména silovému působení a neschodovitým okrajovým podmínkám v LBM. Přirozeným dalším krokem v budoucím výzkumu je rozšířit funkčnost optimalizačního rámce do trojrozměrného prostoru a řešit úlohy v rámci problematiky úplného kavopulmonárního spojení. Pro zajištění funkčnosti optimalizačního rámce ve 3D je zároveň rovněž nutné rozšířit implement	aci interpolačních okrajových do trojrozměrného prostoru. Bez použití interpolačních okrajových podmínek by nebyla zajištěna spojitá závislost hodnot účelové funkce na měnící se použité geometrii. Dále je vhodné prozkoumat, porovnat a implementovat další metody optimalizace, které nevyžadují explicitní předpis účelové funkce. Použití jiných optimalizačních metod může zvýšit kvalitu získaných výsledků a rychlost konvergence. Řadu optimalizačních metod je dále možné efektivně paralelizovat, čímž lze rovněž zkrátit výpočetní čas.		